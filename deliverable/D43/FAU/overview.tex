The problem we are dealing with is called Vehicle Scheduling Problem. It has been dealt with in connection to many applications, including railways, buses and airplanes. 
One of the most important works in the area is the paper by Ahuja et al. \cite{ahuja_solving_2005} who suggest a mixed-integer programming (MIP) formulation of the problem of scheduling locomotives of different type and number. They then use large-scale neighborhood search techniques to come up with good-quality solution. 

Basing on multicommodity flow problem, Cordeau et al. \cite{cordeau_simultaneous_2001} propose an integer programming model for the attribution of locomotives and passenger cars to passenger trips. This is done via a three stage heuristic approach. 
\newline
A more comprehensive overview of the literature can be found in \cite{reuther_schlechte_2018} and \cite{lobel_optimal_1997}. As a general remark - Reuther and Schlechte note in \cite{reuther_schlechte_2018} that the literature in the discussed area is “rather fragmented (in the sense that it is difficult to identify a significant common line [of research])”. They also list a number of reasons for this state of matter.
\newline 
In the latter parts of this document we present two approaches. The first one would be an adaptation of the multi-commodity-flow approach suggested by Ahuja et al. in \cite{ahuja_solving_2005}, whereas the second one would be the novel approach we developed. It differs from the existing approached in the solution time, as well as in a "coarser" treatment of locomotives - we consider them by type rather than by item.

