% Zahlenräume
\newcommand{\F}{\{0, 1\}}
\newcommand{\FR}{[0, 1]}
\newcommand{\N}{\mathbbm{N}}
\newcommand{\Z}{\mathbbm{Z}}
\newcommand{\Q}{\mathbbm{Q}}
\newcommand{\R}{\mathbbm{R}}

\newcommand{\B}{\mathbbm{B}}
\newcommand{\BB}{\mathcal{B}}

\renewcommand{\L}{\mathbbm{L}}
\newcommand{\LL}{\mathcal{L}}

% Landau-Symbol
\newcommand{\OO}{\mathcal{O}}

% Operatoren
\newcommand{\argmain}{\mathrm{\mathop{argmin}}}
\newcommand{\argmax}{\mathrm{\mathop{argmax}}}
%\renewcommand{\vec}{\mathop{vec}}
%\newcommand{\mat}{\mathop{mat}}
\newcommand{\proj}{\mathrm{\mathop{proj}}}
\newcommand{\conv}{\mathrm{\mathop{conv}}}
%\newcommand{\face}[1]{\left\{#1\right\}}
%\newcommand{\prob}{\mathbb P}

\newcommand{\card}[1]{\lvert #1 \rvert}
\newcommand{\abs}[1]{\lvert #1 \rvert}
\newcommand{\norm}[1]{\lVert #1 \rVert}
\newcommand{\lrnorm}[1]{\left\lVert #1 \right\rVert}

%\renewcommand{\binom}[1]{\left(\begin{array}{c} #1 \end{array}\right)}
\newcommand{\choice}[1]{\left\{\begin{array}{rl} #1 \end{array}\right.}
\newcommand{\cvector}[1]{\left(\begin{array}{c} #1 \end{array}\right)}
\renewcommand{\matrix}[2]{\left(\begin{array}{*{#1}{c}} #2 \end{array}\right)}

% Mathematische Akzente
\newcommand{\sol}[1]{\bar{#1}}
\newcommand{\arb}[1]{\hat{#1}}
\newcommand{\nom}[1]{\hat{#1}}
\newcommand{\pert}[1]{\tilde{#1}}
\newcommand{\inc}[1]{\tilde{#1}}
\newcommand{\alt}[1]{\hat{#1}}

% Indexmengen
\newcommand{\iset}[1]{\mathcal{#1}}
\newcommand{\iB}{\iset{B}}
\newcommand{\iT}{\iset{T}}
\newcommand{\iTO}{\alt{\iset{T}}}
\newcommand{\iR}{\iset{R}}
%newcommand{\iP}{\iset{P}}

\newcommand{\set}[1]{\{#1\}}
\newcommand{\lrset}[1]{\left\{#1\right\}}

% Anführungszeichen
\newcommand{\sqm}[1]{`#1'}
\newcommand{\qm}[1]{``#1''}

% Kommentare
\newcommand{\Todo}[1]{\textcolor{red}{#1}}

% Abkürzungen
\newcommand{\ie}{i.e.\ }
\newcommand{\Wlog}{w.l.o.g.\ }
\newcommand{\WLOG}{W.l.o.g.\ }
\newcommand{\Wlogc}{w.l.o.g.}
\newcommand{\WLOGc}{W.l.o.g.}
\newcommand{\cf}{cf.\ }
\newcommand{\eg}{e.g.\ }
\newcommand{\etc}{etc.\ }
\newcommand{\wrt}{w.r.t.\ }

% Fette Tabelleneinträge
\newcommand{\cbf}[1]{\textbf{\boldmath{#1}}}

% Optimierungsprobleme
\newcommand{\st}{\mathrm{s.t.}}
\newcommand{\find}{\mathrm{find}}

\newlength\mysinglespace
\setlength\mysinglespace{0.5\baselineskip}

\newlength\objspace
\setlength\objspace{2\mysinglespace}

\newlength\conspace
\setlength\conspace{\mysinglespace}

\newlength\cconspace
\setlength\cconspace{3\mysinglespace}

\newenvironment{mip}{\alignat{4}}{\endalignat}
\newenvironment{lmip}[1]{\subequations \label{#1} \alignat{4}}{\endsubequations\endalignat}
\newcommand{\objective}[2]{#1\>\>\>\> & #2 \span\span\span\span\span\span}
\newcommand{\stconstraint}[4]{\st\>\>\>\> && #1 && \;\;\;\, #2 & \;\;\;\, #3 && \quad \begin{array}{l} #4 \end{array}}
\newcommand{\constraint}[4]{&& #1 && \;\;\;\, #2 & \;\;\;\, #3 && \quad \begin{array}{l} #4 \end{array}}
\newcommand{\lobjective}[3]{#1\>\>\>\> & #2 \span\span\span\span\span\span \label{#3}}
\newcommand{\lstconstraint}[5]{\st\>\>\>\> && #1 && \;\;\;\, #2 & \;\;\;\, #3 && \quad \begin{array}{l} #4 \end{array} \label{#5}}
\newcommand{\lconstraint}[5]{&& #1 && \;\;\;\, #2 & \;\;\;\, #3 && \quad \begin{array}{l} #4 \end{array} \label{#5}}

% Gleichungsarrays


% Theoremumgebungen
\newtheorem{theorem}{Theorem}[section]
\newtheorem{corollary}[theorem]{Corollary}
\newtheorem{definition}[theorem]{Definition}
\newtheorem{remark}[theorem]{Remark}
\newtheorem{observation}[theorem]{Observation}
\newtheorem{proposition}[theorem]{Proposition}
\newtheorem{example}[theorem]{Example}
\newtheorem{lemma}[theorem]{Lemma}

% Spezifische Makros für diesen Artikel
\newcommand{\PP}{P\xspace}%{$ \mathcal{P} $}
\newcommand{\NP}{NP\xspace}%{$ \mathcal{NP} $}
\newcommand{\GG}{\mathcal{G}}
\newcommand{\VV}{\mathcal{V}}
\newcommand{\EE}{\mathcal{E}}
\newcommand{\II}{\mathcal{I}}
\newcommand{\Gij}{G_{ij}}
\newcommand{\Gji}{G_{ji}}
\newcommand{\Eij}{E_{ij}}
\newcommand{\Sij}{S_{ij}}
\newcommand{\Sji}{S_{ji}}


\newcommand{\CPMC}{(CPMC)\xspace}
\newcommand{\CPMCF}{(CPMCF)\xspace}
\newcommand{\CPMCS}{(CPMCS)\xspace}


% Hyphenations
\hyphenation{ap-pli-ca-tion}
