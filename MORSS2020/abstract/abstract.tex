\documentclass[oneside]{article}
\usepackage[utf8]{inputenc}
\usepackage{authblk}

\title{A shape optimization problem for stationary Navier-Stokes flows in three-dimensional tubes}
\author{Michael Hinterm\"uller}
\author{Axel Kr\"oner}
\author{Hong Nguyen.}
\affil{WIAS Berlin, Germany.}
\date{\today}

\begin{document}
\maketitle
\thispagestyle{empty}
\begin{abstract}
    In order to optimize the shape design of air ducts in combustion engines, we consider a shape optimization problem subject to the stationary Navier-Stokes equations in three dimensions with mixed boundary conditions on domains of polyhedral type. An inflow profile is given at the inlet, a no-slip boundary condition is imposed on the wall, and a no-friction boundary condition on the outlet. To find optimal shapes, we choose a cost functional to achieve a uniform outflow and to minimize the total pressure loss. The associated numerical solution requires an efficient computation and yet accurate approximation of an adjoint-based shape gradient in a shape-gradient-related descent method. We present a numerical example illustrating the method.
\end{abstract}
\textit{Keywords}: Shape optimization, Navier-Stokes equations, adjoint-based method.
\end{document}