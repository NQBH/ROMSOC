
The approach suggested offers railway carriers a different perspective on planning their locomotives. It can well supplement the existing procedures, which focus on making sure that a locomotive is available for a particular client rather than an individual train - our model can suggest possibilities which could not be taken into account otherwise. Once the maintenance constraints are integrated, the approach will also enable planners to better cater to the maintenance plans, by making sure that the locomotive is at the location where it shall undergo its maintenance. 

In our approach, we do not plan empty runs of the locomotives in this model (e.g. locomotives travelling between stations without cargo, just to pick up another train), and such movements of locomotives are required to enable all trains to run. We also assume that drivers need to perform both their first and last job in the planning horizon within a certain set of stations. 

Our further efforts will focus on coupling these findings with a train driver model we work on in parallel to develop a solution which will provide the users with an optimal attribution of both locomotives and drivers. We also work on extending the approach to consider the locomotive maintenance constraints, as well as on empty run scheduling.
