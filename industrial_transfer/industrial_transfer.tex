\documentclass{article}
\usepackage[backend=biber,natbib=true,style=authoryear]{biblatex}
\addbibresource{~/1_PROJECT/Bibliography.bib}
\usepackage[utf8]{inputenc}
\usepackage{graphicx}
\usepackage[colorlinks=true,linkcolor=blue,urlcolor=red,citecolor=magenta]{hyperref}
\usepackage{amsmath,amssymb,amsthm}
\allowdisplaybreaks
\numberwithin{equation}{section}

\title{Knowledge transfer to industrial partners and future valorization:\\
Project 11: Optimal Shape Design of Air Ducts in Combustion Engines}
\author{Michael Hinterm\"uller\thanks{Weierstrass Institute for Applied Analysis and Stochastics, Mohrenstr. 39, 10117 Berlin, Germany.} \and Axel Kr\"oner${}^*$ \and Hong Nguyen${}^*$}
\date{\today}

\begin{document}
\maketitle

The goal of this project is to optimize the shape design of air ducts in combustion engines for the automotive industry. In order to model the flow through a typical air duct, we use the Navier-Stokes equation and then formulate an associated shape optimization problem with suitable objectives subject to the flow, which are widely used in engineering. Additionally, the possible design is restricted by some geometric constraints generated by other components of a combustion engine. The shape gradient of the shape optimization problem will be derived by the continuous adjoint approach. As a benchmark case for this problem, we propose a curved duct geometry with one inlet and one outlet which is inscribed in a design space. The data of this design space has been developed together with engineers from the automotive industry. Furthermore, large Reynolds number flows are known to be unstable and computationally challenging. Thus, appropriate turbulence models will be used in the formulation of the shape optimization problem to overcome these obstacles.

Basically, a finite volume- and/or finite element-based optimization tool which both efficiently and accurately solves appropriate primal and adjoint turbulence models will be implemented and analyzed. The shape gradient of the shape optimization problem established will be calculated based on the complete continuous adjoint approach extending the commonly used frozen turbulence assumption to enhance the accuracy of the sensitivity derivatives. The new solver will be tested for some industry relevant use cases. Together with the industrial partner Math.Tec, we may valorize the software package in the future.


\printbibliography[heading=bibintoc]
\end{document}