For any rail transportation company, scheduling its locomotives lies in the heart of its business. Obviously, no train can move without a locomotive. Those are expensive and hence scarce, therefore their efficient use is a critical success factor for each railway carrier. 
Such a problem can be efficiently solved “by hand” only if the locomotive fleet size is small, at most few machines. When their number is higher, some algorithmic approaches need to be used. Those are often build simply, in such a way that any solution is instantly understandable the planner. On the other hand, such approaches allow for numerous inefficiencies in the locomotive schedules. 
Locomotive scheduling, if posed as an optimization problem, allows the planners / dispatchers not only to speed up the planning process – they can also experiment with various objective functions, resulting in the possibility to quickly generate a number of different schedules for the same set of trains. They usually differ between each other, giving the users the freedom to choose and modify the schedules while taking into account the inputs from other planning challenges existing in the railway world (e.g. driver scheduling, locomotive maintenance scheduling, train scheduling, wagon rotation scheduling etc.). 
